\documentclass{beamer}

\usepackage{cmap}
\usepackage{mathtext}
\usepackage[T2A]{fontenc}
\usepackage[utf8]{inputenc}
\usepackage[english,russian]{babel}
\usepackage{amsmath,amsfonts,amssymb,amsthm,mathtools}
\usepackage{listings}

\title{Physics of Light}

\usetheme{Berlin}

\begin{document}

\begin{frame}
  \titlepage
\end{frame}

\section{Young's Double Slit Experiment}

\begin{frame}[fragile]{Problem}
  Проверить корректность результатов, полученных при моделировании опыта Юнга.
\end{frame}

\begin{frame}[fragile]{Solution}
  Рассмотри аналитическое решение данной задачи и сверим его с полученными на практике результатами.
  \newline
  \newline
  \pause
  Введем следующие обозначения:
  \newline
  \pause
  $a$ — расстояние между щелями
  \newline
  \pause
  $D$ — расстояние между перегородкой и проекционным экраном
  \newline
  \newline
  \pause
  Тогда, при условии того, что $a << D$, геометрическая разность хода $\delta$ в точке $x$
  выражается следующим образом:
  \[\delta = x \dfrac{a}{D}\] 
\end{frame}

\begin{frame}[fragile]{Solution}
  Известно, что яркие полосы - интерференционные максимумы - появляются, когда разность хода
  равна целому числу длин волн.
  \[\delta = p \lambda\]
  \newline
  \pause
  А темные полосы - минимумы - при разности хода, равной нечётному числу полуволн.
  \[\delta = \dfrac{2p + 1}{2 } \lambda\]
  \pause
  где $p \in \mathbb{Z}$, а $\lambda$ - длина волны
\end{frame}

\begin{frame}[fragile]{Solution}
  Выполнив нехитрые преобразования, получаем формулу для расстояния между полосам:
  \[x = \dfrac{D \lambda}{a}\]
\end{frame}

\begin{frame}[fragile]{Solution}
  Параметры нашей модели следующие:
  \newline
  \newline
  \pause
  Размер выводимой интерференционной картины ($S$) - $0,01$м
  \newline
  \pause
  Расстояние между щелями ($a$) - $1 \cdot 10^{-3}$м
  \newline
  \pause
  Расстояние между перегородкой и экраном ($D$) - $2$м
  \newline
  \pause
  Длина волны ($\lambda$) - $400 \cdot 10^{-9}$м
  \newline
  \pause
  Количество яруих полос ($n$) - 14
\end{frame}

\begin{frame}[fragile]{Solution}
  Таким образом, результаты, полученные аналитически:
  \[\dfrac{D \lambda}{a} = 0.0008\]
  \pause
  Результаты нашей модели:
  \[\dfrac{S}{n} \sim 0,000715\]
\end{frame}
  
\end{document}
